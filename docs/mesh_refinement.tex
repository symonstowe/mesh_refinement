% !TEX program = lualatex
% !BIB program = biber
\documentclass[12pt]{iopart}
\usepackage[T1]{fontenc}
\usepackage[utf8]{inputenc}
\usepackage{amssymb}
\usepackage{amsbsy}
\usepackage{graphicx}
\usepackage{color}
\usepackage{subfig}
\usepackage{wasysym}
\graphicspath{{../imgs/}}
% Begin Reference Packages and additions
\usepackage[
backend=biber,
style=authoryear-icomp,
firstinits=true,
doi=false,
isbn=false,
url=false,
citestyle=authoryear,
maxbibnames=9,
maxcitenames=1,
uniquename=false,
uniquelist=false
]{biblatex}
\renewbibmacro*{name:andothers}{% Based on name:andothers from biblatex.def
  \ifboolexpr{
    test {\ifnumequal{\value{listcount}}{\value{liststop}}}
    and
    test \ifmorenames
  }
    {\ifnumgreater{\value{liststop}}{1}
       {\finalandcomma}
       {}%
     \andothersdelim\bibstring[\emph]{andothers}}
    {}}
\addbibresource{refs.bib}
\newcommand{\citeauthorandyear}[2][]{
   \citeauthor{#2} (\citeyear[#1]{#2})}
% End reference packages
\usepackage{multirow}
\usepackage{tikz}
\newcommand{\VB}{{\mbox{\bf V}}}
\newcommand{\YB}{{\mbox{\bf Y}}}
\newcommand{\CB}{{\mbox{\bf C}}}
\newcommand{\JB}{{\mbox{\bf J}}} 
\newcommand{\vB}{{\mbox{\bf v}}}
\newcommand{\COMMENT}[1]{ \textsf{\color{blue}{{COMMENT: #1}}} }
%\graphicspath{{../images/}}
\begin{document}

\title{%
FEM mesh refinement for 3D Electrical Impedance Tomography % Enough acronyms?
}

 \author{%
Symon Stowe$^{1*}$,
Bart\l{}omiej Grychtol$^2$,
Andy Adler$^1$}

\address{
$^1$~Systems and Computer Engineering, Carleton University, Ottawa, Canada
$^2$~\COMMENT{TODO: Fill this in}
}
\ead{*Corresponding author: Email: symonstowe@sce.carleton.ca}
\vspace{10pt}

\begin{abstract}
In this paper we examine the requirement for mesh refinement around 
electrodes in Electrical Impedance Tomography (EIT). In EIT, an analytic solution is
often not possible and the finite
element method (FEM) is used. 
It has been recommended that FEM models be refined around the electrodes, where current
density and sensitivity are highest, but the level of refinement required is poorly understood.
We compare refinement strategies across commonly used meshing software in EIT and
find that, for a fixed number of nodes, error in the sensitivity calculation is minimized 
when the balance point of the nodes is between 70 and 85\% of the tank radius. We also present 
a method of mesh refinement around internal structures and electrodes for EIT.
\end{abstract}

\section{Introduction}
Electrical Impedance Tomography (EIT) reconstructs images of 
electrical tissue properties within a body from electrical
transfer impedance measurements at surface electrodes. For
biomedical imaging applications, it is being actively studied
for monitoring
the movement of air and blood in the thorax, and for imaging
the head and breast. Reconstruction of EIT images requires
the solution of an inverse problem in soft field tomography.
EIT imaging requires an iterative solution in which, at each step,
a sensitivity matrix, $\bf J$, of
the relationship between internal changes and measurements
is calculated, and then a pseudo-inverse of $\bf J$ is used
to update the image estimate. (Several algorithms use
one step of the iterative solution~\parencite{lionheart_eit_2004}.)
EIT image reconstruction is ill-posed, since the physics of
current propagation implies that sensitivity is largest near
the electrodes and smallest in the body centre.

It is therefore clear that a precise  calculation of $\bf J$ is
required for solution accuracy. Since it is generally not
possible to use analytic solutions (because of the non-regular
shapes of biological bodies and the boundary conditions on a
conductive electrode) the finite element method (FEM) is typically
used. 
One key advantage of the FEM is that element size can be
selectively refined in regions to meet solution accuracy. 
The accuracy of the FEM solution will increase as more elements are
added, so a high mesh density is often desired to achieve an 
accurate solution. In EIT the sensitivity is nonuniform 
across the entire model.
Thus
it has generally been recommended in the EIT literature 
that FEMs be refined near electrodes, where the electric 
field and sensitivity are largest~\parencite{adler_electrical_2017}. 
This recommendation gives rise to several questions: 
1) No thorough analysis has been made to determine how much 
refinement is required. Given an ``FEM element budget'', what should
balance of nodes be between the centre of the model and the electrodes?
2) How can mesh refinement be controlled more precisely 
to refine  on arbitrary shapes? 
3) How do different freely available meshing tools that are
commonly used with EIT compare when used to refine 3D meshes?

Previously with EIT, mesh refinement has primarily been either 
constant, or based on 
the complexity of geometric surfaces and lines within a model~\parencite{grychtol_fem_2013}.  
In EIDORS~\parencite{adler_uses_2006} meshes are generated using both 
Netgen~\parencite{schoberl_netgen_1997} and Gmsh~\parencite{geuzaine_gmsh_2009} 
for 2D and 3D models. 
Refinement around electrodes is commonly performed by 
setting a mesh density for the electrodes and allowing the mesh density to 
decay towards the maximum mesh size. This does not allow the user to specify 
the rate of decay or precisely control the mesh size.   

A FEM that accurately represents the anatomy of the imaged region 
can greatly increase the quality of the reconstructed image~\parencite{grychtol_impact_2012},
but increasing the complexity of mesh surfaces presents additional challenges for
mesh refinement.  EIT reconstruction software EIDORS enables users to place electrodes on the surface of complex 
boundaries~\parencite{grychtol_fem_2013}, but the current functionality does not  
enable control of the refinement around the electrodes or internal 
structures.
Most commercially available FEM packages
do not conveniently provide such capability either.

%Due to the increasing number of options for implementing mesh refinement around the
%electrodes it is also important to consider which gives the highest quality mesh. 
%The quality of a mesh dictates the accuracy of the subsequent analysis 
%and solution, and can be characterized by the properties of the tetrahedra
%comprising the model~\parencite{Parthasarathy1993}.

In this paper we present a comparison between Gmsh and 
Netgen based mesh refinement around electrodes and structures, and evaluate the 
effect of mesh refinement techniques on error in  the sensitivity matrix, 
$\bf J$. 
We also present a tool for mesh refinement around both external and internal
electrodes and geometric structures in EIDORS. 

\section{METHODS}
\subsection{Overview}
A cylinder ($\diameter=0.5$~m, height $h=0.25$~m) with four square electrodes 
(5~cm edge length) placed equidistantly around the perimeter at mid-height was
meshed with Netgen (version 5.3.1)~\parencite{schoberl_netgen_1997} and Gmsh 
(version 4.7.0)~\parencite{geuzaine_gmsh_2009}
meshing softwares.
Current was injected between adjacent electrodes and the voltage was measured between the remaining
two electrodes.
For 3D meshes an initial analysis was done building on work from 
Grychtol and Adler~\parencite{grychtol_fem_2013} where mesh density was
set by specifying the maximum edge lengths permitted on electrode surfaces
and in the volume of the FEM.
Results were compared against those generated using ultra-fine meshes. 
Calculations were performed with EIDORS (version 3.10)~\parencite{adler_uses_2006} 
in Matlab 2019b
(The Mathworks, Natick, MA, USA).


\subsection{Mesh Generation}
Meshes of different size were generated with Netgen and Gmsh by manipulating the desired
maximum edge length (maxh parameter) for the entire domain and the electrodes.
Two  mesh analyses were performed. For the first
mesh maximum element lengths were
chosen such as to divide the electrode side of 5~cm into an integer number of
segments of equal size. 
The maximum mesh element length ranged from 1 to 7 subdivisions of the electrode 
edge, while the maximum mesh element length in the ultra-fine reference mesh  
was 15 subdivisions per 
electrode edge. Independent reference meshes were generated for each software.
Two types of models were generated this way. Constant models C1--C7, where the mesh size 
was constant, and refined models R1--R7 where the electrode mesh size was specified and 
dissipated towards an internal mesh element size of 5cm. The numeric value in the mesh 
ID indicated the number of subdivisions per electrode edge. 
In Netgen the mesh decay was not controllable, but in Gmsh the size was set 
to increase evenly from the surface of the electrode to the centre of the model.
Figure~\ref{fig:sample_meshes} shows example meshes of coarse, fine, and refined
meshes. Figure~\ref{fig:electrode_mesh_size} shows the generated mesh structure for 
constant refinement meshes around the electrode 
for both Netgen and Gmsh. 

\begin{figure}
   \includegraphics[width=\columnwidth]{sample_meshes.pdf}
   \caption{\label{fig:sample_meshes} Sample meshes generated with Netgen (top row)
   and Gmsh (bottom row). From left to right: (C1) the coarsest constant
   mesh; (C5) a refined constant mesh; and (R5) a refined mesh with the same
   electrode mesh density as C5 but lower internal mesh density.}
\end{figure}

\begin{figure}
  \includegraphics[width=\columnwidth]{electrode_mesh_size.pdf}
  \caption{\label{fig:electrode_mesh_size} A view of the electrode meshing for all constant-density meshes 
  in Netgen (top row) and Gmsh (bottom row). The figure shows all electrode faces and immediate surrounding
  surroundings from coarsest (C1) to finest (C7). C represents the constant mesh refinement and the number
  represents the specified mesh subdivisions per electrode edge. The reference mesh is equivalent to C15.}
\end{figure}

For the second analysis, the distribution of nodes within the 
model was changed without altering 
the total number of nodes to give M1 -- M17. 
Starting with the constant mesh C3 as M1, the maximum mesh element 
length on the electrode was decreased by 10\% and the maximum mesh size in the centre was
increased so that the total number of elements in the mesh was  within 10\% of the 
original mesh. For mesh M17 the specified electrode refinement was equal to the reference mesh. 
In Netgen the mesh dissipation rate was not further controlled, and in Gmsh the mesh density 
decreased evenly from the electrode surface to the centre of the model. 
To compare these meshes a section of the model was selected
encompassing all points between the centre of the model and a selected electrode face.
The average 
distance, or balance point, along the x-axis of the selected points was expressed 
as a percentage of the tank radius.
This process is illustrated in Figure~\ref{fig:balanceMethods}.
 
\begin{figure}
  \includegraphics[width=\columnwidth]{balance_methods.pdf}
  \caption{\label{fig:balanceMethods} A sketch of the process to determine the 
  balance point of generated meshes. A) the starting mesh; B) nodes between the
  electrode surface and the centre of the model are identified; C) the 
  balance point of the nodes along the x-axis is calculated and indicated 
  by the red plane; D) a histogram showing an example distribution and balance point (red)
  for the selected model.}
\end{figure}

When generating meshes to compare across several mesh density profiles as the balance of the nodes was shifted 
towards the electrodes, 19 meshes were generated for each software including 2 reference meshes. 
Table~\ref{tab:mesh-table} shows the parameters of the resulting odd numbered meshes. 

% Table 
\begin{table}[]
\caption{\label{tab:mesh-table}Mesh parameters for odd numbered meshes generated by Netgen (A) and Gmsh (B) 
to determine the optimal 
node balance. Parameters global maxh and electrode maxh refer to the specified input parameters; the remaining 
columns give parameters from the resulting meshes.}
\input{mesh_table.tex}
\end{table}

%The difference in meshing algorithms meant that the meshes did not have the same number
%of elements, or elements per electrode for a given input. To compare meshing 
%algorithms the error and sensitivity were compared to the average number of mesh elements\
%across all electrodes. 
%
%
%%The settings used to generate all thirty meshes and their sizes are reported
%%in Table~\ref{tab:meshes}. 
%%{\COMMENT{Broke table with new analysis - need to re-insert after electrode mesh fix.}}
%
%The mesh quality was measured as a function of the minimum angle of the tetrahedra
%for the volumetric mesh, and the minimum angle of the triangles in the surface mesh.

\subsection{Simulation}
The potential at each node \VB\ of the mesh was calculated using the finite
element method (FEM) using the linearization 
\begin{equation}
\VB = \YB^{-1}\CB
\end{equation}
where \YB\ is the admittance matrix of the FEM (and a function of conductivity
distribution) and \CB\ is a matrix representing the current injection pattern,
such that \CB$_{ij}$ represents the current injected in electrode $i$ during
the $j$-th stimulation. Here, we drive current of 1~A between two adjacent
electrodes in a single stimulation, so $C = [0\,|\,0\,|\,1\,|\,-1]^T$. 
We pick a node in the centre of the FEM as ground, since it is necessary to
assume the potential on
one node for \YB\ to be invertible~\parencite{Adler1996a}.
We use the complete electrode model and assume contact impedance of
0.01~$\Omega$ in the calculation of the admittance
matrix~\parencite{polydorides_electrode_2002}. 

%The resultant potential distribution in the
%electrode plane, calculated for the finest mesh and subsequently projected onto
%a $512\times512$ pixel grid, is presented in Fig.~\ref{fig:ref:a}.
%The potential distribution \VB\ is used to visualize the current flow
%around the measuring electrodes.
% NOT CURRENTLY LOOKING AT THE CURRENT FLOW! seems to be a repeat of the same stuff

We calculate the sensitivity (or Jacobian) matrix \JB\ of measurements \vB\ to
changes in the conductivity $\sigma$ of individual elements as
$\JB_{ij}=\frac{\partial v_j}{\partial \sigma_i}$ using the adjoint
method~\parencite{polydorides_electrode_2002}. Again, since we only have one measurement, \JB\
is in fact a vector.
We construct a sensitivity image by assigning each element
$i$ of the FEM the value of \JB$_i$ divided by the element's volume.
Mean sensitivity in the plane of electrodes is then calculated by averaging
the sensitivity in fifteen planes parallel to the plane of electrodes and
spanning the height of 5~cm. 
The sensitivity was projected onto a $512\times512$ array and divided into regions
of interest for the centre (C), at the electrode (E) and between the centre and electrode
(M). The resulting sensitivity for the reference mesh calculated with Gmsh and the 
selected regions of interest is presented in Figure~\ref{fig:roiMethods}.


%\subsection{Meshing errors}
%We compare the meshes in terms of the value of the voltage measurement between
%the non-stimulating electrodes, and the mesh quality as determined by the minimum 
%angle in the tetrahedral elements.
% %the distribution of current around the measuring
%%electrodes and the average sensitivity in select regions of interest (ROI) in
%%the electrode plane. 
%%The ROIs are indicated in Fig.~\ref{fig:ref:a}. 
%We use the results obtained
%with the C0 mesh as reference to compare the others against.

\subsection{Electrode refinement for arbitrary FEMs}
Our approach for refinement around electrodes in Gmsh with external electrodes 
also allows for the refinement of arbitrary models with complex structures
such as internal electrodes and tissue boundaries.
A scenario depicting an approximation of 
a probe entering a bone with different 
layers of conductivity. The resulting mesh pictured in 
figure~\ref{fig:adv_mesh}
highlights the ability of this technique
to be used for refinement around electrodes and the control of mesh density  
surrounding internal structures which was previously
very challenging in EIT software.
 
\begin{figure}
  \includegraphics[width=\columnwidth]{advanced_mesh_combined.pdf}
    \caption{\label{fig:adv_mesh} Example FEMs of a probe entering a bone from the surrounding 
    tissue. Refinement is specified around the electrodes and tissue interfaces near the probe.}
\end{figure}
 

\section{Results}

Two analyses of mesh refinement were completed. The first comparing sensitivity error between
meshes with constant refinement and refinement only at the electrodes, and the second 
comparing meshes with different levels of electrode refinement and the same number of nodes.

When comparing constant meshes to meshes with refinement at the electrodes, sensitivity error
was decreased as more nodes were added to the mesh and to the electrodes. The sensitivity
error was lowest in the constant meshes across both Netgen and Gmsh software. Meshes 
generated using Netgen
provided a slightly lower sensitivity error relative to the respective reference mesh
compared to Gmsh, and resulted in meshes with fewer nodes per electrode given the same input
parameters. Figure \ref{fig:results_sens_original} shows the sensitivity error between
constant and refined meshes with respect to the number of nodes per electrode.  

\begin{figure}
  \includegraphics[width=\columnwidth]{sens_error_total.pdf}
  \caption{\label{fig:results_sens_original} Sensitivity error of each mesh as a function 
  of the number of elements per electrode for both Netgen and Gmsh.
\COMMENT{remove the B,C,D,E lines (indicating finer internal mesh size as they get darker) 
- are they adding much other than complexity? - 
also if the axis was number of total nodes could this be more 
interesting? (could have some interessctions?)}}
\end{figure}


Example sensitivity profiles for the M-series meshes are shown in Figure~\ref{fig:roiMethods}.
The resulting sensitivity profile near the electrodes more closely matched the reference case 
when refinement at the electrodes was higher. 

\begin{figure}
  \includegraphics[width=\columnwidth]{roi_methods_figure.pdf}
  \caption{\label{fig:roiMethods} (A) Sensitivity distribution for the reference mesh
  (C15) generated in Gmsh with regions of interest used to compare between models. 
  (B) 3 sensitivity distributions in region $E$ next to the electrodes: (i) Constant mesh M1
  (ii) refined mesh M15 with a balance point of 82\% (iii) reference mesh from (A).
  }
\end{figure}

The total sensitivity error across all meshes is plotted vs. the balance point in
Figure~\ref{fig:balance_sens}. For meshes generated with Gmsh the minimum error was 
achieved when the node balance point was approximately 85\% of the model radius corresponding
to model M15-B, 
and for Netgen generated meshes the minimum sensitivity was achieved in model M13-A at a balance point
of approximately 70\%. Gmsh achieved a lower sensitivity error measured against
the respective reference mesh. For meshes using Netgen refinement, the balance point did not
increase evenly as the electrode density was increased and the internal density decreased. 
To maintain the same number of nodes within the model, Gmsh required a larger internal maxh
than Netgen. Gmsh generated meshes with more nodes for the same input
parameters, but  generally the resulting mesh sizes were closer to those specified. The
resulting mesh parameters for odd numbered meshes can be seen in Table \ref{tab:mesh-table}.
Across all meshes the measurement error when computing the voltage measurements was insignificant
at less than 0.2\%.

\begin{figure}
  \includegraphics[width=\columnwidth]{m-mesh_sens_error.pdf}
  \caption{\label{fig:balance_sens}
  Resulting sensitivity error for Gmsh (blue) and Netgen (red) as the balance of the nodes
  was shifted towards the electrodes.\COMMENT{The green was generted to compare more closely 
  to netgen and can be removed since it is not discussed (I thought it would be better)}}
\end{figure}

  
%  Results show that for both Netgen and Gmsh there is a significant decrease in measurement error 
%  as the number of elements throughout the model is increased. This is shown in Fig.~\ref{fig:results_meas} 
%  where the measurement error on refined meshes decreases as the models are generated with a higher 
%  density of elements. 
%  
%  For both Netgen and Gmsh when using mesh refinement techniques there was no change in the measurement error when compared to 
%  the coarsest models. % THIS DOES NOT MATCH EARLIER RESULTS WHAT IS GOING ON NOW?!?!
%  
%  When comparing mech quality based on the average minimum angle in the tetrahedra, Netgen was shown to have a higher 
%  quality mesh for constant models, but when using refinement Gmsh meshes had a better overall minimum angle score. 
%  These results are presented in Fig.~\ref{fig:results_qual} 
  
\section{Discussion}

We consider several questions on the requirement of FEM refinement in the neighbourhood of
electrodes and the available tools for mesh refinement in EIT. 
1) Given an ``FEM element budget'', what should
balance of nodes be between the centre of the model and the electrodes?
2) How can mesh refinement be controlled more precisely 
to refine  on arbitrary shapes? 
3) How do different freely available meshing tools that are
commonly used with EIT compare when used to refine 3D meshes?

While such refinement is generally agreed to be useful,
we have identified two problems: a lack of systematic analysis of
the required refinement level, and a difficulty in implementing and controlling such
refinement on arbitrary FE models. We present contributions in
both areas.

First, the benefit of electrode refinement has been analyzed by considering
a sequence of refined FEMs compared to a ``gold standard'', uniformly fine
FEM solution. The models were refined either globally or in the electrode
neighbourhood, and the error in the sensitivity matrix $\bf J$ %voltage measurement,  and mesh quality %current distribution and sensitivity
was compared.

%Results are summarized in Fig.~\ref{fig:errors} which indicates
%that model errors near the electrodes decrease equally with electrode
Results are summarized in Figure ~\ref{fig:balance_sens} which indicates
that given a budget of nodes refinement towards the electrodes 
is beneficial in reducing the total sensitivity error. 
The results also indicate that despite the parameters used as input 
to specify mesh density with both programs the result is not totally
controllable by the user. Using the balance point of the nodes towards the electrode on
a model gives a variable that can be optimized to help reduce sensitivity error in
a model. 

It was found that model errors near the electrodes decrease equally with electrode
and uniform refinement. \COMMENT{Just noticed this figure was removed, is figure \ref{fig:roiMethods} enough?}
Model errors deeper in
the body are improved with electrode refinement, but not as much as by
uniform refinement. However, since errors deeper
in the body are so much smaller, this may be less of a factor in many scenarios.

Errors away from the refined areas may be higher 
but with the ability to refine meshes selectively near regions where high sensitivity 
is required this may allow for reduced measurement error while still allowing for quicker 
meshing times.
As more electrodes are added and the model complexity is increased, we expect that 
refinement around the electrodes will continue to reduce total sensitivity error. However 
the node balance analysis will not be possible for irregularly shaped models.

These methods have allowed us to build on mesh refinement strategies
to refine arbitrary surface meshes on regions surrounding electrodes~\parencite{grychtol_fem_2013},
to include refinement on internal structures or electrodes, and  have more control over this
refinement.  
These additions fill an important need in EIT to allow for more accurate models 
of regions surrounding internal structures and electrodes, and 

% Based on these results, we produced an additional model combining the electrode
% refinement of R8 and the maximum element size of C5 (18947 nodes and 98595
% elements). Its sensitivity
% error near the electrode was better than C2, while deeper in the medium it was
% on par with C5.  It produced measurement error of only 0.11~mV.


\section{Conclusion}

In summary, as expected, refinement of  meshes near electrodes 
does improve model accuracy in terms of sensitivity.
We recommend that, for each EIT imaging case, required model accuracy be
determined from an analysis of the system, 
and to minimize the sensitivity error in the forward solution 
the balance of the nodes should be approximately 
85\% towards the radius of the
model when the dissipation of mesh refinement is constant. 

\printbibliography

\end{document}
